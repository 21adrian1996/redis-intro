\section{Redis}
\subsection{Charakterisierung}
Redis ist ein Open Source, In-Memory Datenbanksystem, das als vorallem als Datenbank, Cache und Message Broker verwendet werden kann.\\
Redis ist BSD lizenziert und stellt somit keine Anforderungen an die Weiterverteilung der Software.\\
Der Name Redis entstand als Abkürzung für remote dictionary server.\\
Entwickelt wurde Redis 2009 von Salvatore Sanfilippo in C und gehört mitlerweile zu VMWware. Die aktuelle Version (Stand 5. November 2019) ist 5.0.5, wobei die aktuelle LTS Version 3.2.11 ist.
Redis ist single-threaded und lässt sich einfach replizieren.

\subsubsection{Platformen, welche Redis verwenden}
Redis erfreut sich grosser beliebtheit und ist laut DB-Engines.com die verbreitetste Key-Value Datenbank.
Einige der grossen Kunden sind etwa Twitter, Github, Snapchat und StackOverflow.
https://redis.io/topics/whos-using-redis
\subsection{Architektur anhand von Redis}
\subsection{Installation}
Server und Client
\subsection{Datentypen}
https://redis.io/topics/data-types
\subsection{Abfragesprache}
https://www.cheatography.com/tasjaevan/cheat-sheets/redis/
\subsubsection{Transaktionen}
Transkaktionsverhalten - Theoretisch und Testszenario (Demo in der Präsentation)
https://redis.io/topics/transactions
\subsection{Benchmarks}
https://redis.io/topics/benchmarks
\subsection{Persistenz}
Concurrency Verhalten, Serialisierbarkeit
https://redis.io/topics/persistence
\subsection{Replikation}
Verteilte Datenhaltung, Skalierbarkeit, Protokolle, Architektur
https://redis.io/topics/replication
\subsection{Optimierungsmöglichkeiten}
\subsection{GUI}
\subsection{API}
mit Demo
