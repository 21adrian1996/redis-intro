\section{Key Value Store}
\subsection{Definition}
Ein Key Value Store basiert auf einer Tabelle mit genau zwei Spalten: \\
In der einen befindet sich ein eindeutiges Identifikationsmerkmal, der Schlüssel (Key), in der anderen der Wert (Value).
Somit wird jeder Wert eindeutig einem Schlüssel zugewiesen.
Welche Datentypen im Wert gespeichert werden können, ist abhängig vom verwendeten Key-Value Store.
Es handelt sich somit um eine NoSQL Datenbank.
\subsection{Vorteile und Nachteile}
\subsubsection{Vorteile}
\begin{itemize}
	\item Geschwindigkeit
	\item Skalierbarkeit
	\item Simples Modell, einfach verständlich
\end{itemize}
\subsubsection{Nachteile}
\begin{itemize}
	\item Für Daten mit vielen relationalen Abhängigkeiten sehr komplex
	\item Keine Lookup Optimierungen für das Suchen
\end{itemize}
\subsection{Einsatzgebiete}
Überall wo schnelle Zugriffszeiten bei großen Datenmengen benötigt werden, eignen sich Key Value Stores. Typische Einsatzgebiete sind deshalb Warenkörbe in Onlineshops oder das Speichern von Session-Daten.

\subsection{In-Memory}
Bei vielen Key Value Stores handelt es sich um In-Memory Datenbanken. Im Gegensatz zu herkömmlichen On-Disk Datenbanken verwenden diese anstelle des ROM, den Arbeitspeicher eines Computers. Da die Schreib- und Lesezugriffe auf HDDs (durch die magnateische Scheiben) und SSDs (als Flash-Speicher) beschränkt sind, können in herkömmlichen On-Disk Datenbanken längere Ladezeiten auftretten. Durch die Verwendung von In-Memory, kann dieses Limit stark erhöht werden. Nachteil davon ist, dass die Daten nicht direkt persistent gespeichert sind.
Dadurch sind In-Memory Datenbanken gut geeignet, um grosse Datenmenegen auszuwerten.
\clearpage
\subsection{Top 5 Key Value Stores}
Gemäss db-engines.com (Stand 27.10.2019) sind die folgenden Datenbanksysteme führend:
\begin{enumerate}
\item Redis
\item Amazon DynamoDB
\item Microsoft Azure Cosmos DB
\item Memcached
\item Hazelcast
\end{enumerate}
